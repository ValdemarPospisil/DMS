\documentclass[a4paper,12pt]{article}
\usepackage[czech]{babel}
\usepackage[utf8]{inputenc}
\usepackage[T1]{fontenc}
\usepackage{geometry}
\usepackage{graphicx}
\usepackage{hyperref}
\usepackage{booktabs}
\geometry{margin=2.5cm}

\begin{document}

\begin{center}
    {\Large \textbf{TECHNICKÁ ZPRÁVA}} \\
    \vspace{0.5cm}
    {\large \textbf{Analýza výkonnosti databázového systému}} \\
    \vspace{0.3cm}
    Interní dokument společnosti DataSystems s.r.o. \\
    \vspace{0.2cm}
    Verze 1.2 - květen 2024
\end{center}

\vspace{1cm}

\section{Úvod}
Tento dokument obsahuje výsledky analýzy výkonnosti databázového systému PostgreSQL v produkčním prostředí. Analýza byla provedena na základě požadavku vedení IT oddělení z důvodu zhoršující se odezvy systému během špičkového zatížení.

\section{Metodologie testování}
Pro účely testování byl použit nástroj pgBench, který umožňuje simulovat různé typy zátěže na databázový systém. Testování probíhalo v izolovaném prostředí s následující konfigurací:

\begin{itemize}
    \item Server: Dell PowerEdge R740, 2x Intel Xeon Gold 6248R, 384 GB RAM
    \item Operační systém: Ubuntu Server 22.04 LTS
    \item PostgreSQL verze: 15.2
    \item Testovací data: 500 GB (kopie produkčních dat)
\end{itemize}

\section{Výsledky měření}
Výkonnostní testy byly zaměřeny na tyto klíčové oblasti:
\begin{itemize}
    \item Propustnost systému při různých počtech současných připojení
    \item Doba odezvy na komplexní dotazy
    \item Využití systémových prostředků během zátěže
\end{itemize}

\subsection{Propustnost systému}
\begin{table}[h]
\centering
\begin{tabular}{@{}lrrr@{}}
\toprule
\textbf{Počet klientů} & \textbf{Transakce/s} & \textbf{Latence [ms]} & \textbf{Využití CPU [\%]} \\
\midrule
10 & 352 & 28 & 25 \\
50 & 1245 & 40 & 58 \\
100 & 1876 & 53 & 75 \\
200 & 2134 & 94 & 87 \\
500 & 2210 & 226 & 98 \\
\bottomrule
\end{tabular}
\caption{Výsledky testů propustnosti}
\end{table}

\subsection{Identifikované problémy}
Na základě provedených testů byly identifikovány následující problémy:
\begin{enumerate}
    \item Nedostatečná velikost sdílené paměti (shared\_buffers)
    \item Neoptimální nastavení write-ahead log (WAL)
    \item Chybějící indexy na často používaných dotazech
    \item Nevhodná strategie vacuum
\end{enumerate}

\section{Navrhovaná řešení}
Pro zlepšení výkonnosti doporučujeme provést následující změny v konfiguraci:

\begin{itemize}
    \item Zvýšení parametru shared\_buffers na 25\% dostupné RAM (96 GB)
    \item Optimalizace WAL parametrů:
    \begin{itemize}
        \item wal\_level = replica
        \item max\_wal\_size = 16GB
        \item min\_wal\_size = 4GB
    \end{itemize}
    \item Vytvoření indexů pro kritické dotazy (viz příloha A)
    \item Nastavení agresivnějšího autovacuum:
    \begin{itemize}
        \item autovacuum\_vacuum\_scale\_factor = 0.05
        \item autovacuum\_analyze\_scale\_factor = 0.02
    \end{itemize}
\end{itemize}

\section{Dopad změn}
Po implementaci navrhovaných změn v testovacím prostředí došlo k následujícím zlepšením:
\begin{itemize}
    \item Zvýšení propustnosti o 45\% při 200 současných připojeních
    \item Snížení latence o 62\% při špičkovém zatížení
    \item Snížení využití CPU při stejném zatížení o 15-20\%
\end{itemize}

\section{Závěr}
Na základě provedených testů a analýz doporučujeme implementovat navrhované změny v produkčním prostředí. Před implementací je však nutné provést zálohu systému a naplánovat údržbové okno s minimálním dopadem na uživatele.

\vspace{1cm}

\begin{flushright}
Zpracoval: Ing. Tomáš Veselý \\
Datum: 4. 5. 2024
\end{flushright}

\end{document}
