\documentclass{beamer}

\usepackage[utf8]{inputenc}
\usepackage[czech]{babel}
\usepackage{graphicx}
\usepackage{hyperref}
\usepackage{listings}
\usepackage{xcolor}

\usetheme{boxes}
\usecolortheme{lily}

\title{Paperless-ngx}
\subtitle{Moderní systém pro správu dokumentů}
\author{Valdemar Pospíšil}
\institute{UJEP - PřF}
\date{\today}

\lstset{
  basicstyle=\ttfamily\small,
  keywordstyle=\color{blue},
  commentstyle=\color{green!50!black},
  stringstyle=\color{red},
  breaklines=true,
  showstringspaces=false
}

\begin{document}

\begin{frame}
  \titlepage
\end{frame}

\begin{frame}{Obsah}
  \tableofcontents
\end{frame}

\section{Úvod do Paperless-ngx}

\begin{frame}{Co je Paperless-ngx?}
  \begin{itemize}
    \item Open-source systém pro správu dokumentů
    \item Fork původního projektu „Paperless“
    \item Určen pro digitalizaci, organizaci a vyhledávání dokumentů
    \item Umožňuje skenování, OCR, indexování, třídění, štítkování
    \item Moderní webové rozhraní a REST API
  \end{itemize}
  \begin{figure}
    \centering
    \includegraphics[width=0.35\textwidth]{paperless-logo.png}
  \end{figure}
\end{frame}

\begin{frame}{Historie a vývoj}
  \begin{itemize}
    \item 2015: Vznik projektu Paperless (Daniel Quinn)
    \item 2020: Fork Paperless-ng – modernizace
    \item 2021+: Paperless-ngx – aktivní komunita, časté aktualizace
    \item Podpora Dockeru, OCR, REST API, pravidel pro třídění dokumentů
  \end{itemize}
\end{frame}

\section{Architektura a funkce}

\begin{frame}{Architektura systému}
  \begin{itemize}
    \item Backend: Python (Django)
    \item Frontend: Angular/TypeScript
    \item Databáze: PostgreSQL / MariaDB / SQLite
    \item Redis: fronta úloh
    \item Celé běží v Docker kontejnerech
  \end{itemize}
\end{frame}

\begin{frame}{Klíčové funkce Paperless-ngx}
  \begin{itemize}
    \item OCR (Tesseract) – převod naskenovaného textu
    \item Automatická klasifikace dokumentů podle pravidel
    \item Fulltextové vyhledávání a indexace obsahu
    \item Podpora tagů, metadata, spotřeba (účtenky)
    \item REST API pro integraci
    \item Uživatelé a oprávnění
    \item Export, import, zálohování
  \end{itemize}
\end{frame}

\section{Pro koho je Paperless-ngx vhodný}

\begin{frame}{Cílová skupina}
  \begin{itemize}
    \item Jednotlivci – digitalizace domácích dokumentů
    \item Malé firmy – archivace smluv, faktur, objednávek
    \item IT nadšenci – automatizace, skriptování pomocí API
    \item Instituce – interní správa dokumentů, bezpečné úložiště
  \end{itemize}
\end{frame}

\section{Výhody a nevýhody}

\begin{frame}{Výhody}
  \begin{itemize}
    \item Zdarma a open-source
    \item Aktivní vývoj, dokumentace
    \item Jednoduché nasazení přes Docker
    \item Webové rozhraní přístupné z více zařízení
    \item Automatizace pomocí pravidel a API
  \end{itemize}
\end{frame}

\begin{frame}{Nevýhody}
  \begin{itemize}
    \item Nutnost vlastní správy a hostingu
    \item Složitější na začátku pro ne-IT uživatele
    \item Omezené možnosti sdílení (např. ne jako Google Drive)
    \item Nutné zálohování a údržba databáze
  \end{itemize}
\end{frame}

\section{Nasazení pomocí Docker Compose}

\begin{frame}[fragile]{Ukázka Docker Compose}
  \begin{lstlisting}
version: "3.4"
services:
  paperless:
    image: ghcr.io/paperless-ngx/paperless-ngx
    restart: always
    ports:
      - 8000:8000
    volumes:
      - ./data:/usr/src/paperless/data
      - ./media:/usr/src/paperless/media
    environment:
      PAPERLESS_REDIS: redis://redis:6379
      PAPERLESS_DBHOST: db
  redis:
    image: redis
    restart: always
  db:
    image: postgres
    restart: always
    environment:
      POSTGRES_DB: paperless
      POSTGRES_USER: paperless
      POSTGRES_PASSWORD: paperless
  \end{lstlisting}
\end{frame}

\section{Závěr}

\begin{frame}{Shrnutí}
  \begin{itemize}
    \item Paperless-ngx je výkonný DMS pro domácí i firemní použití
    \item Umožňuje automatickou správu a vyhledávání dokumentů
    \item Nasazení pomocí Dockeru je jednoduché a rychlé
    \item Ideální pro IT nadšence a malé firmy
  \end{itemize}
\end{frame}

\end{document}
